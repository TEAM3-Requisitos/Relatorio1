“Entender os requisitos de um problema está entre as tarefas mais difíceis enfrentadas por um engenheiro de software”.(PRESSMAN, 2006, p.126).

Tendo em vista a afirmação acima, infere-se que é necessário que a engenharia de requisitos forneça mecanismos apropriados para entender as necessidades dos clientes, analisar viabilidade, especificar a solução sem ambiguidades, validar, gerenciar as necessidades conforme suas alterações.

Com base nas observações acima, este relatório tem como objetivo descrever de forma detalhada todos os mecanismos utilizados para levantar, especificar e detalhar todos os requisitos de um software, bem como os mecanismos utilizados nas fases de especificar solução, modelar processo, validar e gerenciar os requisitos.


\section {Visão Geral do Relatório}

Este relatório abrange os seguintes tópicos:

\begin{itemize}
\item Contextualização do Cliente: Conhecer a empresa, entender o problema e Descreve o contexto de negócio.
\item Abordagem de engenharia de requisitos e Justificativa: Esclarecimento da abordagem escolhida e os motivos que levaram a equipe a escolher a metodologia.
\item Processo de engenharia de requisitos: Descreve o processo que será realizado para o levantamento e documentação dos requisitos.
\item Técnicas de elicitação de requisitos: Descreve as técnicas de elicitação de requisitos que serão utilizadas no trabalho 2.
\item Gerenciamento de requisitos: Descreve os processos que serão realizados para o gerenciamento de requisitos e as ferramentas utilizada para esse objetivo.
\item Planejamento do Projeto: Descreve o planejamento das atividades e as ferramentas
\end{itemize}

\section {Contexto do Cliente}

A empresa Eletrojun possui um sistema de compartilhamento de projetos como ilustrado no apendice '' ”, esse sistema tem objetivo de reunir diversos estudantes no intuito de compartilhar projetos e ideias.
O software desenvolvido pela Eletrojun está incompleto, há diversas funcionalidades que não estão implementadas, além de erros que devem ser corrigidos. Esse sistema tem um papel muito importante para a empresa, que é o papel de unir todos os estudantes da Universidade de Brasília - Campus FGA, no intuito de desenvolver, criar, melhorar e divulgar projetos. A empresa espera que esse software possa ser o canal de comunicação entre os estudantes, de modo que, projetos possam ser desenvolvidos com o auxílio desse sistema
Esse site tem funcionalidades para gestão de outros projetos e necessita de uma ferramenta para compartilhamento de arquivos, esse sistema deve permitir que o usuário compartilhe ideias com outros usuários.
Atualmente a Eletrojun não possui uma ferramenta para o gerenciamento de ideias, e sofre com diversos problemas administrativos, no que tange ao
compartilhamento dessas ideias.
A estrutura da Eletrojun é vertical, e segue uma hierarquia bem definida, esse tipo de estrutura pode ser favorável ao bom andamento do projeto, visto que, a equipe de Analista de Requisitos terá que se reunir apenas com diretores da empresa.
