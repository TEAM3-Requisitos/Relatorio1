O gerenciamento de requisitos é um processo que consome muitos recursos. neste processo deve-se decidir sobre:

1 - Identificação de requisitos: Cada requisito deve ser identificado unicamente de modo que possa ser feita a referência cruzada entre este e outros requisitos para que ele possa ser utilizado nas avaliações de rastreabilidade.
2 - Processo de gerenciamento de mudanças: É o conjunto de atividades que avaliam o impacto e custo das mudanças.
3 - Políticas de rastreabilidade: Essas políticas definem os relacionamentos entre os requisitos e o projeto do sistema, que devem ser registrados, e como estes registros devem ser mantidos.
3 - Apoio de Ferramentas: O gerenciamento de requisitos envolve o processamento de grandes quantidades de informações sobre os requisitos. As ferramentas que podem ser usadas variam desde sistemas especializados de gerenciamento de requisitos a planilhas  e sistemas simples de banco de dados.
(SOMMERVILLE, 2007, p.108).

\section{Atributos de Requisitos}

Requisitos não são constituídos apenas pela especificação do que é requerido, mas também por um conjunto de informações adicionais que auxiliam a interpretar e gerenciar os requisitos. (25)

Buscando uma melhor forma de gerenciar, foram definidos os atributos que cada requisito deverá conter, e são eles:

1 - Origem

Os requisitos possuirão um código único que será sua identificação.

TABELAAAAA


2 - Status

Irá indicar o grau de completude do requisito, podendo ser:
Completo: requisito já foi implementado no sistema.

\begin{itemize}
\item Em progresso: requisito está sendo implementado no sistema.
\item Não iniciado: indica que o requisito está no backlog mas não foi implementado ainda.
\end{itemize}

3 - Prioridade

Este atributo indicará o nível de importância para os stakeholders, sistema e outros requisitos.

\begin{itemize}
\item Alta: quando o requisito é de suma importância para os stakeholders ou quando sem ele o sistema não funciona.
\item Média: requisito que é importante para os interessados, mas que ainda sem ele o sistema funciona de forma básica.
\item Baixa: requisito que não compromete o funcionamento do sistema, podendo ser uma funcionalidade opcional.
\end{itemize}

4 - Complexidade

Indica nível de dificuldade para implementar o requisito em questão.

\begin{itemize}
\item Alta: requisito com grau de dificuldade elevado, sendo necessário um grande esforço da equipe para implementação.
\item Média: requisito com grau de dificuldade médio, sendo necessário um esforço da equipe mas de forma moderada.
\item Baixa: requisito com baixo grau de dificuldade de  implementação.
\end{itemize}

5 - Risco

Requisitos que apresentam alguma possibilidade de risco para o sistema durante o desenvolvimento.

\begin{itemize}
\item Alto: requisito com grande possibilidade de risco, necessita de uma atenção maior da equipe.
\item Médio: requisito com média possibilidade de risco.
\item Baixo: requisito com baixa possibilidade de risco.
\end{itemize}

Os requisitos seguirão este padrão:

TABELAAA
