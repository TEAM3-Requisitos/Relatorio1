A palavra vai explicar a funcionalidade do levantamento de requisitos, apesar de ser um termo mais generalista, ela vai na essência do levantamento de requisitos.
Essa parte é a que o profissional de TI, que geralmente desempenha o papel de engenheiro de requisitos, conversa com o cliente para entender sua necessidade, e o mesmo o orienta, para que suas necessidades sejam atendidas.
É importante ressaltar que essa é uma fase muito crucial para todos os projetos, principalmente na área de software, pôs nesse momento é que a solução do problema do cliente vai ser elaborada.

\section {Técnicas de elicitação de requisitos}

Ao analisarmos o nosso contexto para o levantamento de requisitos, percorremos e analisamos as diversas técnicas que são usadas atualmente, como o levantamento orientado a ponto de vista, etnografia, prototipagem, entrevista, questionário, brainstorming e JAD. Logo após analisarmos chegamos a conclusão que a técnica que se adéqua melhor ao nosso contexto é a entrevista.
Partindo do princípio que já existe uma parte do nosso software pronto, e que nossa cliente conhece bem os requisitos e tem noções de desenvolvimento de software, técnicas como JAD apesar de promover a cooperação, o trabalho em grupo e o entendimento, acabam não se adaptando, devido sua metodologia ser baseada na criação e na resolução de problemas, onde os desenvolvedores ajudam os usuários do sistema a criarem problemas e eles mesmos proporem soluções.
O Brainstorming é uma técnica que a princípio suas ideias não parecem convencionais, mas são encorajadas, estimulando frequentemente os participantes para a elaboração de soluções criativas. Mas a exploração de novas ideias não se aplica ao nosso caso, porque a ideia do nosso software já existe.
Já a técnica da aplicação de questionários é geralmente usada quando existe uma distância física entre os usuários, não se aplicando ao nosso caso, já que nossa cliente tem uma proximidade física do nosso grupo, e aplicação dos questionários se torna obsoleta e desnecessária.
A prototipagem parte do pressuposto de fazermos protótipos para a validação do cliente, implementando os aspectos críticos do sistema, mas nessa caso já temos um protótipo já validado pelo cliente, técnica que se adotada, geraria um retrabalho.
As técnicas de Workshops são usadas para promover a interação entre os stakeholders, não se aplicando novamente ao nosso contexto, sendo que nos dirigiremos a apenas uma representante da Eletrojun, apesar.

\subsection {Entrevista}

A entrevista visa identificar o problema, propor a solução, negociar os aspectos de abordagem, identificar um conjunto preliminar  de requisitos do projeto.

\begin{itemize}
\item As entrevistas serão conduzidas com a participação dos engenheiros de software e outros interessados.
\item A agenda de entrevistas deve possibilitar uma entrevista que cubra todos os pontos importantes,  e abra espaço para o fluxo de novas ideias.
\item Deve haver  um controle da reunião através de um plano de entrevista, para que não haja uma dispersão do conteúdo pertinente ao entrevistador.
\end{itemize}
