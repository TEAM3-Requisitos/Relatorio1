Visto que a abordagem ágil requer uma proximidade com o cliente, e a mesma tem uma disponibilidade de tempo para que possamos fazer reuniões semanais, o grupo chegou a conclusão que a abordagem ágil se adequa melhor ao nosso contexto.
Além da proximidade com a cliente, devido a mesma ser uma estudante da mesma faculdade dos componentes do grupo, em nossa primeira reunião, ela demonstrou possuir conhecimentos da área de software, que foi um dos fatores determinantes para nossa escolha, deixando os componentes mais seguros quanto a metodologia adotada.

Outro fator importante analisado, foi que ela já orientou outro aluno da faculdade que era da Eletrojun para fazer uma parte do sistema, demonstrando uma certa experiência na descrição de requisitos, outro fator analisado e que contou como ponto positivo para a definição da abordagem.
